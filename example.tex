%%
%% This is file `example.tex',
%% generated with the docstrip utility.
%%
%% The original source files were:
%%
%% lcd.dtx  (with options: `example')
%% 
%% Copyright (c) 2004 Mike Kaufmann, all rights reserved
%% 
%% This program is provided under the terms of the
%% LaTeX Project Public License distributed from CTAN
%% archives in directory macros/latex/base/lppl.txt.
%% 
%% Author: Mike Kaufmann
%%         Mike.Kaufmann@ei.fh-giessen.de
%% 
%% \CharacterTable
%%  {Upper-case    \A\B\C\D\E\F\G\H\I\J\K\L\M\N\O\P\Q\R\S\T\U\V\W\X\Y\Z
%%   Lower-case    \a\b\c\d\e\f\g\h\i\j\k\l\m\n\o\p\q\r\s\t\u\v\w\x\y\z
%%   Digits        \0\1\2\3\4\5\6\7\8\9
%%   Exclamation   \!     Double quote  \"     Hash (number) \#
%%   Dollar        \$     Percent       \%     Ampersand     \&
%%   Acute accent  \'     Left paren    \(     Right paren   \)
%%   Asterisk      \*     Plus          \+     Comma         \,
%%   Minus         \-     Point         \.     Solidus       \/
%%   Colon         \:     Semicolon     \;     Less than     \<
%%   Equals        \=     Greater than  \>     Question mark \?
%%   Commercial at \@     Left bracket  \[     Backslash     \\
%%   Right bracket \]     Circumflex    \^     Underscore    \_
%%   Grave accent  \`     Left brace    \{     Vertical bar  \|
%%   Right brace   \}     Tilde         \~}
%%
\documentclass[a4paper]{article}
\usepackage[latin1]{inputenc}
\usepackage{color}
\usepackage{lcd}

\parindent0pt
\parskip1ex plus.3ex minus.2ex
\pagestyle{empty}
 
\newcommand\lcd{\textLCD{3}|LCD|}
\newcommand\bs{\char '134 }
\definecolor{lightgreen}{rgb}{0.05,0.97,0.55}
\definecolor{darkgreen}{rgb}{0.22,0.26,0.19}
\definecolor{lightblue}{rgb}{0.9,0.91,0.99}
\definecolor{darkblue}{rgb}{0.14,0.2,0.66}
\definecolor{lightred}{rgb}{1.0,0.27,0.37}
\definecolor{darkred}{rgb}{0.37,0.14,0.18}

\DefineLCDchar{euro}{00111010001111101000111110100000111}

\begin{document}
\centerline{\textbf{\LARGE Some Examples for the \lcd\ package.\footnote{The
source of this example file is part of \texttt{lcd.dtx}.}}}

As seen in the headline and here, the \lcd\ package calculates the size for
LCD-text in normal text (\verb|\textLCD|) automaticly. It works for all
fontsizes:

\begin{center}
{\tiny MM M \lcd\ M MM tiny}\hfill{\Huge Huge MM M \lcd\ M MM}

{\scriptsize MM M \lcd\ M MM scriptsize}\hfill{\huge huge MM M \lcd\ M MM}

{\footnotesize MM M \lcd\ M MM footnotesize}\hfill{\LARGE LARGE MM M \lcd\ M MM}

{\small MM M \lcd\ M MM small}\hfill{\Large Large MM M \lcd\ M MM}

{\normalsize MM M \lcd\ M MM normalsize}\hfill{\large large MM M \lcd\ M MM}
\end{center}

Now let's have some colored
\LCDcolors{darkgreen}{lightgreen}\textLCD[0]{8}|LCD-text|.
Here first the colors where set with
\verb|\LCDcolors{darkgreen}{lightgreen}|\footnote{The color names where
defined with \texttt{\bs definecolor} from the \textsf{color} package in the
preamble.}
and then the LCD-text where done with \verb+\textLCD[0]{8}|LCD-text|+.
To invert the LCD, just exchange the
\LCDcolors{lightgreen}{darkgreen}\textLCD[0]{6}|colors|
(\verb|\LCDcolors{lightgreen}{darkgreen}|).

\begin{minipage}[t]{.5\textwidth}
Now some seperate LCD representations.  But first let's  change the colors
to some not as ugly. The LCD was generated with
\begin{verbatim}
\LCD{4}{18}|LCD representation|
           |made with the LCD |
           |package for LaTeX |
           |04.01.2004 {clock} 18:23|
\end{verbatim}
\end{minipage}
\hspace{\fill}
\begin{minipage}[t]{.46\textwidth}
\mbox{}

\LCDcolors{darkblue}{lightblue}%
\LCD{4}{18}|LCD representation|
           |made with the LCD |
           |package for LaTeX |
           |04.01.2004 {clock} 18:23|
\end{minipage}

The \verb|{clock}| is a so called multi-letter character. It generates the
clock symbol.

As you can see, there is a black colored frame around it. The frame color
can be changed with the optional first argument of \verb|\LCDcolors|
(\verb|\LCDcolors[red]|\ldots; left part of figure 1). And
with \verb|\LCDnoframe| you can disable frames (reenabled with
\verb|\LCDframe|; right part of figure 1).
Of course \verb|\LCD| works within a figure environment.

\begin{figure}[h]
\LCDcolors[red]{darkblue}{lightblue}%
\LCD{4}{18}|LCD representation|
           |made with the LCD |
           |package for LaTeX |
           |04.01.2004 {clock} 18:45|
\hspace{\fill}\LCDnoframe
\LCD{4}{18}|LCD representation|
           |made with the LCD |
           |package for LaTeX |
           |04.01.2004 {clock} 18:47|
\caption{Example with red colored frame and without frame}
\end{figure}

\LCDcolors[lightgreen]{lightred}{darkred}\LCDframe
\LCD{2}{36}|For more information please refer to|
           |the documentation!                  |
\end{document}
\endinput
%%
%% End of file `example.tex'.
